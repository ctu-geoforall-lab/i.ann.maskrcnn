%%%%%%%%%%%%%%%%%%%%%%%%%%%%%%%%%%%%%%%%%
% Beamer Presentation
% LaTeX Template
% Version 1.0 (10/11/12)
%
% This template has been downloaded from:
% http://www.LaTeXTemplates.com
%
% License:
% CC BY-NC-SA 3.0 (http://creativecommons.org/licenses/by-nc-sa/3.0/)
%
%%%%%%%%%%%%%%%%%%%%%%%%%%%%%%%%%%%%%%%%%

%----------------------------------------------------------------------------------------
%	PACKAGES AND THEMES
%----------------------------------------------------------------------------------------

\documentclass{beamer}

\mode<presentation> {

% The Beamer class comes with a number of default slide themes
% which change the colors and layouts of slides. Below this is a list
% of all the themes, uncomment each in turn to see what they look like.

%\usetheme{default}
%\usetheme{AnnArbor}
%\usetheme{Antibes}
%\usetheme{Bergen}
%\usetheme{Berkeley}
%\usetheme{Berlin}
%\usetheme{Boadilla}
%\usetheme{CambridgeUS}
%\usetheme{Copenhagen}
%\usetheme{Darmstadt}
%\usetheme{Dresden}
%\usetheme{Frankfurt}
\usetheme{Goettingen}	% vpravo
%\usetheme{Hannover}
%\usetheme{Ilmenau}
%\usetheme{JuanLesPins}
%\usetheme{Luebeck}
%\usetheme{Madrid}
%\usetheme{Malmoe}			
%\usetheme{Marburg}
%\usetheme{Montpellier}
%\usetheme{PaloAlto}
%\usetheme{Pittsburgh}
%\usetheme{Rochester}
%\usetheme{Singapore}			
%\usetheme{Szeged}
%\usetheme{Warsaw}

% As well as themes, the Beamer class has a number of color themes
% for any slide theme. Uncomment each of these in turn to see how it
% changes the colors of your current slide theme.

%\usecolortheme{albatross}
%\usecolortheme{beaver}
%\usecolortheme{beetle}
%\usecolortheme{crane}
%\usecolortheme{dolphin}
%\usecolortheme{dove}
%\usecolortheme{fly}
%\usecolortheme{lily}			
%\usecolortheme{orchid}
%\usecolortheme{rose}
%\usecolortheme{seagull}
%\usecolortheme{seahorse}
%\usecolortheme{whale}
%\usecolortheme{wolverine}

%\setbeamertemplate{footline} % To remove the footer line in all slides uncomment this line
%\setbeamertemplate{footline}[page number] % To replace the footer line in all slides with a simple slide count uncomment this line

%\setbeamertemplate{navigation symbols}{} % To remove the navigation symbols from the bottom of all slides uncomment this line
}

\usepackage[utf8]{inputenc}	% kódování textu
\usepackage[english]{babel}		% zavedení češtiny
\usepackage{amsmath,amsfonts,amssymb}	% matematika
\usepackage{graphicx} % Allows including images
\usepackage{booktabs} % Allows the use of \toprule, \midrule and \bottomrule in tables
\usepackage{multirow}	% slouceni radek v tabulce
\usepackage{multicol}	% slouceni sloupcu v tabulce
\usepackage{longtable}	% rozdeleni tabulky pres vice stran
\usepackage{enumerate}	% seznamy
\usepackage{float}
\usepackage{lscape}		% stranka na sirku
\usepackage{fancyhdr}
\usepackage{url}
\usepackage{array}
\usepackage{subfigure}
\usepackage{dirtree}
\usepackage{setspace}
\usepackage{color}
\usepackage{listings}
\usepackage{multimedia}
\usepackage{tikz}
\usepackage{fancyvrb}


%------------------------------------------------------------------
%	TITLE PAGE
%------------------------------------------------------------------

\title[]{Mask R-CNN in GRASS GIS} % The short title appears at the bottom of every slide, the full title is only on the title page

\author{Ondřej Pešek} % Your name
\institute[ČVUT] % Your institution as it will appear on the bottom of every slide, may be shorthand to save space
{
Czech Technical University in Prague \\ % Your institution for the title page
%\medskip
Faculty of Civil Engineering \\
%\medskip
Department of Geomatics
}
\date{29. 8. 2019} % Date, can be changed to a custom date
\titlegraphic{\includegraphics[width=1.5cm]{pictures/logo2.pdf}}

\begin{document}

\begin{frame}
\titlepage % Print the title page as the first slide
\end{frame}

\begin{frame}
\frametitle{Table of contents} % Table of contents slide, comment this block out to remove it
\tableofcontents % Throughout your presentation, if you choose to use \section{} and \subsection{} commands, these will automatically be printed on this slide as an overview of your presentation
\end{frame}

%------------------------------------------------------------------
%	PRESENTATION SLIDES
%------------------------------------------------------------------

%------------------------------------------------------------------
\section{Motivation} % Sections can be created in order to organize your presentation into discrete blocks, all sections and subsections are automatically printed in the table of contents as an overview of the talk
%------------------------------------------------------------------

\begin{frame}

\frametitle{Motivation}

Situation
\begin{itemize}
	\item higher density of the satellite monitoring systems
	\item higher density of aerial imagery
	\item vectorization of analogue maps
	\visible<2->{\item higher quality}
	\visible<3->{\item open data}
	\visible<4->{\item data standardization}
\end{itemize}

\end{frame}

%------------------------------------------------------------------

\begin{frame}

\frametitle{Motivation}

Common classification methods
\begin{itemize}
	\item manual classification
	\visible<2->{\item supervised classification}
	\visible<3->{\item unsupervised classification}
\end{itemize}

\bigskip

GRASS GIS
\begin{itemize}
	\item manual classification
	\begin{itemize}
		\item GRASS Digitizing tool
	\end{itemize}
	\visible<2->{\item supervised classification
	\begin{itemize}
		\item g.gui.iclass + i.maxlik
	\end{itemize}}
	\visible<3->{\item unsupervised classification
	\begin{itemize}
		\item i.cluster + i.maxlik
	\end{itemize}}
\end{itemize}

\end{frame}

%------------------------------------------------------------------

\begin{frame}

\frametitle{Motivation}

Why neural networks?
\begin{itemize}
	\item human brain is the most powerful tool we know
	\item we are trying to get human-understandable results
\end{itemize}

\begin{figure}[ht]
	\includegraphics<1>[height=0.45\textheight]{pictures/nn1.png}
	\includegraphics<2>[height=0.44\textheight]{pictures/nn2.png}
	\includegraphics<3>[height=0.45\textheight]{pictures/nn3.jpg}
	\caption{Zdroj: [1]}
\end{figure}

\end{frame}

%------------------------------------------------------------------

\section{Theoretical framework}

\subsection{Convolutional neural networks} % A subsection can be created just before a set of slides with a common theme to further break down your presentation into chunks

%------------------------------------------------------------------

\begin{frame}

\frametitle{Convolutional neural networks}

\begin{figure}[ht]
	\includegraphics[width=0.6\textwidth]{pictures/conv.jpg}
\end{figure}

\visible<2->{Why convolutional neural networks?
\begin{itemize}
	\item ResNet got in ILSVRC 2016 top 5 error of 3.6 \%
	\item<3> human 8 \%
\end{itemize}

Source: [1]}

\end{frame}

%------------------------------------------------------------------

\subsection{Mask R-CNN}

\begin{frame}

\frametitle{Mask R-CNN}

\visible<2>{Instance segmentation}

\begin{figure}[ht]
	\includegraphics<1>[width=0.9\textwidth]{pictures/segmentations.png}
	\includegraphics<2>[width=0.65\textwidth]{pictures/instance-segmentation.png}
	\caption{Source: [2]}
\end{figure}

\end{frame}

%------------------------------------------------------------------

\begin{frame}

\frametitle{Mask R-CNN}

Two parts:
\begin{itemize}
	\item backbone
	\item head
\end{itemize}

\begin{figure}[ht]
	\includegraphics[width=0.8\textwidth]{pictures/maskrcnn.png}
	\caption{Source: [3]}
\end{figure}

\end{frame}

%------------------------------------------------------------------

\begin{frame}

\frametitle{Mask R-CNN}

Backbone architecture:
\begin{itemize}
	\item ResNet
	\item<2-> RPN
\end{itemize}

\begin{figure}[ht]
	\includegraphics<1>[height=0.4\textheight]{pictures/bottleneck-block.jpg}
	\includegraphics<2>[height=0.4\textheight]{pictures/fasterrcnn.png}
	\includegraphics<3>[height=0.4\textheight]{pictures/fasterrcnn-anchors.png}
	\caption{Source: [3]}
\end{figure}

\end{frame}

%------------------------------------------------------------------

\begin{frame}

\frametitle{Mask R-CNN}

Head architecture:
\begin{itemize}
	\item softmax $\rightarrow$ class
	\item<2-> regression $\rightarrow$ bounding box
	\item<3-> FCN $\rightarrow$ mask
\end{itemize}

\begin{figure}[ht]
	\includegraphics<1-2>[height=0.3\textheight]{pictures/fastrcnn.png}
	\includegraphics<3>[height=0.3\textheight]{pictures/maskrcnn-head.png}
	\caption{Source: [4]}
\end{figure}

\end{frame}

%------------------------------------------------------------------

\section{Implementation}

%------------------------------------------------------------------

\subsection{Usage}

\begin{frame}

\frametitle{Usage}

Workflow:
\begin{itemize}
	\item i.ann.maskrcnn.train
	\item i.ann.maskrcnn.detect
\end{itemize}

\end{frame}

%------------------------------------------------------------------

\subsection{i.ann.maskrcnn.train}

\begin{frame}

\frametitle{i.ann.maskrcnn.train}

Workflow behind
\begin{itemize}
	\item<1-> configuration of the model
	\item<2-> feed the model with pre-trained weights
	\item<3-> read the training dataset
	\item<4-> train
	\item<5-> save the model
\end{itemize}

\begin{figure}[ht]
	\includegraphics<1>[width=.7\textwidth]{pictures/gui-train-conf.png}
	\includegraphics<2>[width=.7\textwidth]{pictures/gui-train-weights.png}
	\includegraphics<3>[width=.7\textwidth]{pictures/gui-train-dataset.png}
	\includegraphics<4>[width=.7\textwidth]{pictures/gui-train-train.png}
	\includegraphics<5>[width=.7\textwidth]{pictures/gui-train-save.png}
\end{figure}

\end{frame}

%------------------------------------------------------------------

\subsection{i.ann.maskrcnn.detect}

\begin{frame}

\frametitle{i.ann.maskrcnn.detect}

Workflow behind
\begin{itemize}
	\item<1-> load the model
	\item<2-> detection for each raster
	\item<3-> vectorization
\end{itemize}

\begin{figure}[ht]
	\includegraphics<1>[width=.7\textwidth]{pictures/gui-detect-model.png}
	\includegraphics<2>[width=.7\textwidth]{pictures/gui-detect-images.png}
	\includegraphics<3>[width=.7\textwidth]{pictures/gui-detect-vectorize.png}
\end{figure}

\end{frame}

%------------------------------------------------------------------

\subsection{Results}

\begin{frame}

\frametitle{Results}

\begin{figure}[ht]
	\only<1>{
	\includegraphics[width=.9\textwidth]{pictures/out1.png}
	\caption{loss function 0.96, 54000 training images}}
	\only<2>{
	\includegraphics[width=.9\textwidth]{pictures/out2.png}
	\caption{loss function 0.96, 54000 training images}}
	\only<3>{
	\includegraphics[width=.9\textwidth]{pictures/out3.png}
	\caption{loss function 0.96, 54000 training images}}
	\only<4>{
	\includegraphics[width=.9\textwidth]{pictures/out_b_1.png}
	\caption{epoch 1, loss function 35.01, 2400 training images}}
	\only<5>{
	\includegraphics[width=.9\textwidth]{pictures/out_b_10.png}
	\caption{epoch 10, loss function 5.87, 2400 training images}}
	\only<6>{
	\includegraphics[width=.9\textwidth]{pictures/out_b_50.png}
	\caption{epoch 50, loss function 1.36, 2400 training images}}
	\only<7>{
	\includegraphics[width=.9\textwidth]{pictures/out_b_150.png}
	\caption{epoch 150, loss function 0.63, 2400 training images}}
	\only<8>{
	\includegraphics[width=.9\textwidth]{pictures/out_b_180.png}
	\caption{epoch 180, loss function 0.50, 2400 training images}}
\end{figure}

\end{frame}

%------------------------------------------------------------------

\section{Conclusion}

\begin{frame}[fragile]

\frametitle{Conclusion}

\begin{itemize}
	\item source code
	\begin{itemize}
		% \item \url{https://github.com/ctu-geoforall-lab-projects/dp-pesek-2018}
		\item \url{https://github.com/ctu-geoforall-lab/i.ann.maskrcnn}
		\item \url{https://github.com/OSGeo/grass-addons/tree/master/grass7/imagery/i.ann.maskrcnn}
	\end{itemize}
	\item installation using command \verb|g.extension extension=i.ann.maskrcnn|
	\item next steps
	\begin{itemize}
		\item multispectral rasters
		\item training on rasters and vectors imported in GRASS
		\item more architectures
	\end{itemize}
\end{itemize}

\end{frame}

%------------------------------------------------------------------

\section{Sources}

\begin{frame}

\frametitle{Sources}

[1] RUSSAKOVSY, Olga et al.
ImageNet Large Scale Visual Recognition
Challenge. International Journal of Computer Vision IJCV. 2015, 115, n. 3,
pp. 211–252.

[2] http://cs231n.stanford.edu/

[3] HE, Kaiming et al. Mask R-CNN. In: International Conference on Computer
Vision (ICCV). 2017.

[4] GIRSHICK, Ross. Fast R-CNN. In: International Conference on Computer
Vision (ICCV). 2015.

\end{frame}

%------------------------------------------------------------------

\begin{frame}

\centerline{Thank you for your attention.}

\end{frame}

%------------------------------------------------------------------
\iffalse
\section{Reakce na otázky oponenta}

\begin{frame}

\frametitle{Reakce na otázky oponenta}

Jaké jsou výhody/nevýhody užití neuronových sítí namísto klasických postupů?

\begin{center}
	\noindent\makebox[\linewidth]{\rule{0.9\textwidth}{0.4pt}}
\end{center}

\bigskip

Výhody:
\begin{itemize}
	\item<2-> přesnost
	\item<3-> minimalizovaná potřeba vytvářet ad hoc řešení
	\item<4-> obecnost
\end{itemize}

Nevýhody:
\begin{itemize}
	\item<5-> výpočetní náročnost
	\item<6-> časová náročnost
	\item<7-> potřeba rozsáhlých trénovacích dat
	\item<8-> odvážným štěstí nepřeje vždy
\end{itemize}

\end{frame}

%------------------------------------------------------------------

\begin{frame}

\frametitle{Reakce na otázky oponenta}

Implementoval jste modifikace využitého software i zpět do zdrojů?

\begin{center}
	\noindent\makebox[\linewidth]{\rule{0.9\textwidth}{0.4pt}}
\end{center}

\bigskip

Ano:
\begin{itemize}
	\item<2-> GRASS GIS
\end{itemize}

Ne:
\begin{itemize}
	\item<3-> GRASS GIS
	\item<4-> Matterport, Inc.
	\item<7-> potřeba rozsáhlých trénovacích dat
	\item<8-> odvážným štěstí nepřeje vždy
\end{itemize}

\end{frame}

%------------------------------------------------------------------

\begin{frame}

\frametitle{Reakce na otázky oponenta}

Shledáváte některé části svého kódu natolik obecnými, aby mohly být využiti i při vývoji dalších podobných nástrojů?

\begin{center}
	\noindent\makebox[\linewidth]{\rule{0.9\textwidth}{0.4pt}}
\end{center}

\bigskip

\visible<2->{Ano.}

\end{frame}

%------------------------------------------------------------------

\begin{frame}

\frametitle{Reakce na otázky oponenta}

V textu práce jste zmínil a rozebíral podezřelé chování modulů skýtajících lepší výsledky při vyšší ztrátové funkci. Můžete tento případ ještě rozvést?

\begin{center}
	\noindent\makebox[\linewidth]{\rule{0.9\textwidth}{0.4pt}}
\end{center}

\bigskip

Možné příčiny:
\begin{itemize}
	\item<2-> přeoptimalizace
	\item<3-> lokální odchylka
	\item<4-> nedostatečná data
\end{itemize}

\begin{figure}
	\centering
	\begin{minipage}{.45\textwidth}
		\centering
		\begin{figure}[ht]
	  		\includegraphics[width=4.5cm]{pictures/out_b_150.png}
		\end{figure}
    \end{minipage}%
    \begin{minipage}{.6\textwidth}
		\centering
		\begin{figure}[ht]
			\includegraphics[width=4.5cm]{pictures/out_b_180.png}
		\end{figure}
	\end{minipage}
\end{figure}

\end{frame}
\fi
%==================================================================
\end{document} 

